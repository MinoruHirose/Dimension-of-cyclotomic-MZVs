
\documentclass[oneside,reqno]{amsart}
\usepackage[T1]{fontenc}
\usepackage[latin9]{inputenc}
\usepackage{geometry}
\geometry{verbose,tmargin=3.5cm,bmargin=3.5cm,lmargin=2.5cm,rmargin=2.5cm}
\usepackage{enumitem}
\usepackage{bm}
\usepackage{amsthm}
\usepackage{amssymb}
\usepackage{stmaryrd}

\makeatletter
%%%%%%%%%%%%%%%%%%%%%%%%%%%%%% Textclass specific LaTeX commands.
\numberwithin{equation}{section}
\numberwithin{figure}{section}
\newlength{\lyxlabelwidth}      % auxiliary length
\theoremstyle{plain}
\newtheorem{thm}{\protect\theoremname}
\theoremstyle{definition}
\newtheorem{defn}[thm]{\protect\definitionname}
\theoremstyle{plain}
\newtheorem{prop}[thm]{\protect\propositionname}

%%%%%%%%%%%%%%%%%%%%%%%%%%%%%% User specified LaTeX commands.
\usepackage{shuffle}


\providecommand{\definitionname}{Definition}
\providecommand{\propositionname}{Proposition}
\providecommand{\theoremname}{Theorem}

\begin{document}
\title{Implementation details}

\maketitle
%
\global\long\def\biseq#1#2{\left\llbracket \begin{array}{c}
#1\\
#2
\end{array}\right\rrbracket }%
\global\long\def\biseqp#1#2{\left\llbracket \begin{array}{c}
#1\\
#2
\end{array}\right\rrbracket _{p}}%


\subsection{Notation}
\begin{itemize}
\item For the tuple $\bm{a}=(a_{1},\dots,a_{m})$ and $\bm{b}=(b_{1},\dots,b_{n})$,
the concatenation $(a_{1},\dots,a_{m},b_{1},\dots,b_{n})$ is denoted
by $\bm{a}\cdot\bm{b}$.
\item Fix an positive integer $N$.
\end{itemize}

\subsection{The set $X^{\otimes d}$}

Let $\mathsf{X}_{d}$ be the set of the formal sum of tuples $(u_{1},\dots,u_{d},l_{1},\dots,l_{d})\in\{0,1,\dots,N-1\}^{d}\times\mathbb{Z}_{\geq0}^{d}$.
We identify $X^{\otimes d}$ and $\mathsf{X}_{d}$ by

\[
\biseq{\zeta_{N}^{u_{1}},\dots,\zeta_{N}^{u_{d}}}{l_{1},\dots,l_{d}}\mapsto(u_{1},\dots,u_{d},l_{1},\dots,l_{d}).
\]


\subsection{The map $D_{d}^{\mathrm{iter}}$}
\begin{defn}
The map
\[
\mathsf{push}:\mathbb{Z}_{\geq0}^{d}\times\{0,\dots,d-2\}\times\mathbb{Z}_{\geq0}\to\mathbb{Z}_{\geq0}^{d}
\]
 is defined by
\[
\mathsf{push}((l_{1},\dots,l_{d}),i,r)=(l_{1},\dots,l_{i},l_{i+1}+l_{i+2}-r,l_{i+3},\dots,l_{d})
\]
for $i\geq0$.
\end{defn}

\begin{defn}
The map
\[
\mathsf{Diterpre}:(\mathbb{Z}/N\mathbb{Z})^{e}\times\mathbb{Z}_{\geq0}^{e}\times(\mathbb{Z}/N\mathbb{Z})^{d}\times\mathbb{Z}_{\geq0}^{d}\to\mathsf{X}_{d}
\]
is defined by
\begin{align*}
 & \mathsf{Diterpre}(\bm{v}=(v_{1},\dots,v_{e}),\bm{m}=(m_{1},\dots,m_{e}),\bm{u}=(u_{1},\dots,u_{d}),\bm{l}=(l_{1},\dots,l_{d}))\\
 & =\mathsf{Diterpre}(\bm{v}\cdot(u_{1}-u_{2}),\bm{m}\cdot(l_{1}),(u_{2},\dots,u_{d}),(l_{2},\dots,l_{d}))\\
 & \quad+\sum_{i=2}^{d-1}\sum_{r=l_{i}}^{l_{i-1}+l_{i}}(-1)^{r-l_{i}}\binom{r}{l_{i}}\mathsf{Diterpre}(\bm{v}\cdot(u_{i}-u_{i+1}),\bm{m}\cdot(r),(u_{1},\dots,u_{i-1},u_{i+1},\dots,u_{d}),\mathsf{push}(\bm{l},i-2,r))\\
 & \quad-\sum_{i=1}^{d-1}\sum_{r=l_{i}}^{l_{i}+l_{i+1}}(-1)^{l_{i}}\binom{r}{l_{i}}\mathsf{Diterpre}(\bm{v}\cdot(u_{i+1}-u_{i}),\bm{m}\cdot(r),(u_{1},\dots,u_{i},u_{i+2},\dots,u_{d}),\mathsf{push}(\bm{l},i-1,r))\\
 & \quad+\sum_{r=l_{d}}^{l_{d-1}+l_{d}}(-1)^{r-l_{d}}\binom{r}{l_{d}}\mathsf{Diterpre}(\bm{v}\cdot(u_{d}),\bm{m}\cdot(r),(u_{1},\dots,u_{d-1}),\mathsf{push}(\bm{l},d-2,r)).
\end{align*}
for $d\geq2$ and
\[
\mathsf{Diterpre}((v_{1},\dots,v_{e}),(m_{1},\dots,m_{e}),(u),(l))=(v_{1},\dots,v_{e},u,m_{1},\dots,m_{e},l).
\]
\end{defn}

\begin{defn}
The map
\[
\mathsf{Diter}:(\mathbb{Z}/N\mathbb{Z})^{d}\times\mathbb{Z}_{\geq0}^{d}\to\mathsf{X}_{d}
\]
is defined by
\[
\mathsf{Diter}(\bm{u},\bm{l})=\mathsf{Diterpre}((),(),\bm{u},\bm{l}).
\]
\end{defn}

\begin{prop}
Under the identification $X^{\otimes d}\simeq\mathsf{X}_{d}$, $D_{d}^{\mathrm{iter}}((\bm{u},\bm{l}))\in Y^{\otimes d}$
coincides with the image of $\mathsf{Diter}(\bm{u},\bm{l})\in X^{\otimes d}$
in $Y^{\otimes d}$.
\end{prop}


\subsection{The kernel $X^{\otimes d}\to Y^{\otimes d}$}
\begin{prop}
Under the identification $X^{\otimes d}\simeq\mathsf{X}_{d}$, the
kernel of $X^{\otimes d}\to Y^{\otimes d}$ is spanned by the following
elements:
\begin{itemize}
\item $(u_{1},\dots,u_{d},l_{1},\dots,l_{d})$ for $u_{1},\dots,u_{d}\in\{0,\dots,N-1\}^{d}$
and $l_{1},\dots,l_{d}\in\mathbb{Z}_{\geq0}$ with $(u_{i},l_{i})=(0,0)$
for some $i$.
\item $(u_{1},\dots,u_{d},l_{1},\dots,l_{d})-(-1)^{l_{i}}(u_{1},\dots,u_{i-1},-u_{i},u_{i+1},\dots,u_{d},l_{1},\dots,l_{d})$
for $u_{1},\dots,u_{d}\in\{0,\dots,N-1\}^{d}$, $l_{1},\dots,l_{d}\in\mathbb{Z}_{\geq0}$
and $i\in\{1,\dots,d\}$
\item $(u_{1},\dots,u_{d},l_{1},\dots,l_{d})-M^{l_{i}}\sum_{s=0}^{M-1}(u_{1},\dots,u_{i-1},u_{i}/M+sN/M,u_{i+1},\dots,u_{d},l_{1},\dots,l_{d})$
for $u_{1},\dots,u_{d}\in\{0,\dots,N-1\}^{d}$, $l_{1},\dots,l_{d}\in\mathbb{Z}_{\geq0}$,
$i\in\{1,\dots,d\}$ and divisor $M$ of $N$ with $u_{i}\equiv0\pmod{M}$
and $(u_{i},l)\neq(0,0)$. (We can omit the caes $M=1$).
\end{itemize}
\end{prop}

\begin{thebibliography}{1}
\bibitem{key-2}M. Hirose, On the motivic fundamental group of the
multiplicative group minus $N$-th roots of unity, in preparation.

\end{thebibliography}

\end{document}
